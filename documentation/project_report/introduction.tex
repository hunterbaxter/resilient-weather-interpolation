\section{Introduction}
The motivation for doing this project was to demonstrate an understanding of many of the tools and frameworks used in 'Principles of Cloud Computing' to solve an interesting theoretical problem in an aesthetically pleasing manner.
The hypothetical problem is that one wants to collect data over a large geographical region,
but due to some constraints,
one cannot collect data in a tightly packed grid of points or polygons over the area.
As a result, to get dense data,
one has to use interpolation strategies to fill the sparsity.
The title of the project, resilient weather interpolation,
comes from the emphasis on reliable design and sound software engineering practice.

% TODO cite https://insights.sei.cmu.edu/blog/system-resilience-what-exactly-is-it/
A system is defined as resilient if it continues to carry out its mission in the face of adversity \cite{systemResilience}.
The reason the project focuses on resiliency is that real world systems in the tech industry must be able to endure unexpected adverse conditions, 
and in some cases, even adversarial attacks. 
In such a small scale project, one cannot attempt nor claim to have built a fully resilient system,
but it changes the lens through one views software.

To continue with the theme of the class,
the authors wanted to use tools they would encounter, or use themselves,
in the tech industry. 
Since the majority of the authors first job out of university is a software engineering position,
the development cycle was a point of focus. 
'DevOps', a compounded term of development and operations,
is defined as an outline of software development practices and organizational culture that speeds up the deliver of high quality software through automation and the integration of development and IT operations teams \cite{devops}.
Although the scope of the project is small,
and there are not two disjoint development and operations teams, 
one can still have 'DevOps' influences. 
The reasoning is that if team members buy into the concept,
a team can effectively simulate a larger organization, 
and as a result deliver a better project despite the small academic setting. 