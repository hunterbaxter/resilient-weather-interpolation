% NOTE: This section needs most references, theoretical citations, etc.
\section{Architecture}
\subsection{Overview}
% Meghan
Our objective when designing the system architecture for this project was to utilize many of the technologies and frameworks we have discussed throughout the semester.
With that in mind, we decided to approach each portion of our system as a micro-service which could be easily containerized and managed by a container orchestration service.
In doing so, we have designed a system which is both resilient and highly available. 

\subsection{Producer}
% Meghan
Much of this project relies on geo spatial data.
Namely, we utilized live weather date served from a data vendor Weatherbit.io.
Our producer makes use of a background scheduler which periodically polls weatherbit's API to retrieve the current weather.
These polls occur every 15 minutes (the minimum possible interval for which we are guaranteed to receive new data).
In order to make this data available to the rest of our services, we utilized Kafka.
We published this data to the Kafka brokers, encoding the topics so that they reflect the location of the weather station from which they were received.
This service is containerized in a Docker container and hosted on its own AWS EC2 instance.

\subsection{Broker}
% hb
\cite{kreps2011kafka}
\cite{hunt2010zookeeper}

\subsection{RESTful API}
\cite{fielding2002principled}
% TODO: mention interpolation theory: hb


\subsection{Web Application}
% Meghan
To provide a meaningful visualization of the interpolated data, we created a React web application which utilizes Kepler.gl to display the weather in the region of interest.
This service runs inside of a Docker container on an AWS EC2 instance and is managed by our container orchestration service. 

\subsection{CI/CD}
% Meghan
One of our goals for this project was high availability. In order to achieve this, we needed to implement a CI/CD pipeline. We utilized Jenkins, an open source DevOps tool specifically designed for managing CI/CD pipelines. If any new changes are observed on any branch, the code is built and tested to ensure that the new version is successful before sending it to production.

\subsection{Container Orchestration}
% hb